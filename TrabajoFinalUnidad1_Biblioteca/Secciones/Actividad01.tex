
\section{Desarrollo del Trabajo} 

\begin{enumerate}[1.]
         

	\item Resumen

  Para el presente trabajo se tomo referencia el como esta estructurada la base de datos de la bibilioteca universitaria, se diseño la base datos con las clases requeridas para el trabajo y posteriormente se analiz\'o base de datos para encontrar la tabla de hechos en este caso se considero como tabla de hechos a la tabla RESERVA, como segundo se identific\'o la granulidad y las tablas dimensionales, luego se realizar\'on las consultas respectivas para hallar los indicadores requeridos.
\\
	\item Objetivo

 El objetivo principal de este trabajo es apoyar a las consultas de los usuarios finales en un almacén de datos. Se orienta entorno a la comprensibilidad y rendimiento para obtener informcaci\'on objetiva para tomar las desiciones con el menor porcentaje de errores.
\\	
	\item Listado de Indicadores
	
-  Porcentaje de profesores y estudiantes del progrma de utiliza los recursos bibliograficos disponibles en la escuela de sistemas, por curso y ciclo

-  porcentaje de referencias  bibliograficas de los trabajos de investigacion de la escuela de sistemas que utiliza los recursos bibliograficos disponibles

- Porcentaje de referencias  bibliograficas  de los silabos de los cursos de la escuela  de sistemas que utiliza recurdsos bibliograficos.
\\	
	
	\item Diseño de Modelo multidimensional
	
- Modelo proyectado de la Base de Datos:
	\begin{center}
	\includegraphics[width=17cm]{./Imagenes/BD_biblioteca} 
	\end{center}

- ModeloDimensional de la Base de Datos:
	\begin{center}
	\includegraphics[width=17cm]{./Imagenes/bdDimensional} 
	\end{center}

- Consultas propuestas:
\\
1. Query: select *, ((select count(codigo) from ListaM where tipo='11') * 100) / (SELECT count(codigo) FROM ListaM) as porcen from ListaM

2. Query: Para calcular el porcentaje de uso del material bibliografico de la bibliotecca necesitamos registrar los trabajos de investigacion de cada curso y de estos registrar el material bibliografico que se indica en el documento, para luego buscar si se encuentran en la biblioteca\\

select (SELECT  count(matBibliograficoId)  FROM trabajosInvestigacion.matBibliografico
INNER JOIN Libros ON trabajosInvestigacion.matBibliografico.nombre = Libros.nombre) * 100 / (select count(*) from Libros) as Porcentaje\\

3. Query: Para calcular el porcentaje de uso del material bibliografico de la bibliotecca necesitamos registrar los silabos de cada curso y de estos registrar el material bibliografico que se indica, para luego buscar si se encuentran en la biblioteca\\

Cantidad de material bibliografico que esta en los libros de la biblioteca\\
SELECT matBibliograficoId, matBibliografico.nombre
FROM matBibliografico
INNER JOIN Libros ON matBibliografico.nombre = Libros.nombre\\
\\
Porcentaje material bibliografico que esta en los libros de la biblioteca\\
select (SELECT  count(matBibliograficoId)  FROM matBibliografico
INNER JOIN Libros ON matBibliografico.nombre = Libros.nombre) * 100 / (select count(*) from Libros) as Porcentaje asdsadsa


	\item Bibliograf\'ia
\\
- https://es.wikipedia.org/wiki/Modelado\_dimensional
\\
- https://blog.bi-geek.com/modelo-dimensional/
\\
- https://searchdatacenter.techtarget.com/es/definicion/Base-de-datos-multidimensional-MDB
\\
- https://es.calameo.com/books/002299301667571c7ab05
\\
-https://searchdatacenter.techtarget.com/es/consejo/Tablas-de-dimension-vs-tablas-de-hechos-Cual-es-la-diferencia


\end{enumerate} 
